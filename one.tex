\documentclass{article}
\usepackage[utf8]{inputenc}

\title{Comp 3400 Assignment 1 Solution}
\author{Rehnuma Tarannum (100838870)}
\date{7 February 2016}

\begin{document}

\maketitle


\begin{enumerate}
    \item 
    \begin{description}
    \item[(a)] Knowledge base
    \begin{itemize}
        \item A - Argentina is coloured Blue
        \item B- Bolivia coloured Blue
        \item C-Chile is coloured Blue
        \item P -Peru coloured Blue
    \end{itemize}
    \item [(b)] The Equations below list the scenarios when we try to colour the map with Argentina, Peru, Bolivia and Chile with two colours.
    \begin{itemize}
        \item $ a \rightarrow \neg c \wedge \neg b \wedge p$.
        \item $ \neg a \rightarrow \ c \wedge \ b \wedge \neg p$.
        \item $ b \rightarrow \neg a \wedge \neg c \wedge \neg p$.
        \item $ \neg b \rightarrow a \wedge c \wedge p$.
        \item $ c \rightarrow \neg a \wedge \neg b \wedge \neg p$.
        \item $ \neg c \rightarrow a \wedge b \wedge p$.
        \item $ p \rightarrow \neg c \wedge \neg b \wedge a$.
        \item $ \neg p \rightarrow \ c \wedge \ b \wedge \neg a$
    \end{itemize}
    \item [(c)] The input file and the Prover9 output file
     \begin{verbatim}
        b-> -c & -a & -p.
        -b-> c & a & p.
        c -> -b & -a & -p.
        -c -> b & a & p.
        p -> -c & -b & a.
        -p -> c & b & -a.
        
        ==============================prooftrans==================
        Prover9 (32) version Dec-2007, Dec 2007.
        Process 1826 was started by rehnumatarannum on dhcp-44-252.cu-wireless-faculty.carleton.ca,
        Thu Feb  4 12:54:52 2016
        The command was "/Applications/Prover9-Mace4-v05B.app/Contents/Resources/bin-mac-intel/prover9".
        ========================= end of head=====================

        ================== end of input ==========================

        ==================== PROOF================================

        % -------- Comments from original proof --------
        % Proof 1 at 0.00 (+ 0.00) seconds.
        % Length of proof is 4.
        % Level of proof is 2.
        % Maximum clause weight is 1.
        % Given clauses 0.

        9 F # label(non_clause) # label(goal).  [goal].
        13 F.  [assumption].
        23 -F.  [deny(9)].
        24 $F.  [copy(23),unit_del(a,13)].

    ===================== end of proof ==========================

    \end{verbatim}
    \item[(d)] If we tried to colour the map with 3 colours, as long as the combination countries P-B-C (Peru, Bolivia and Chile) and A-B-C (Argentina, Bolivia and Chile) have different colours respectively it is possible to colour the map with 3 colours since Argentina and Peru don't share a border. In any other case colouring the map with 3 different colours would fail, since adjacent countries share a border.
    \end{description}
    \item
    \begin{description}
    \item [(a)] Knowledge Base
    \begin{itemize}
        \item p - label: “The diamond is not here”
        \item q - label: “Exactly one of the label is true ”
        \item d - The diamond is in the gold box
    \end{itemize}
    \item[(b)] Come up with logic equations from the propositional knowledge, feed it to Prover9 and give its goal the output you expect.
    \begin{verbatim}
        Assumptions:        
        -p | -q.			
        p & -d.
        q & -p
        
        Goal:
        -d
    \end{verbatim}
    \item[(c)]Prover9 proves by contradiction so, it first assumed that my assumption given to it where wrong and then it went about try to prove it. During this process if it found any contingencies it declares the original assumption to be true.
    
    \begin{verbatim}
    ====================prooftrans===============================
    Prover9 (32) version Dec-2007, Dec 2007.
    Process 1682 was started by rehnumatarannum on dhcp-44-252.cu-wireless-faculty.carleton.ca,
    Thu Feb  4 11:28:11 2016
    The command was "/Applications/Prover9-Mace4-v05B.app/Contents/Resources/bin-mac-intel/prover9".
    ====================== end of head===========================

    =================== end of input ============================

    =================== PROOF ===================================

    % -------- Comments from original proof --------
    % Proof 1 at 0.00 (+ 0.00) seconds.
    % Length of proof is 5.
    % Level of proof is 2.
    % Maximum clause weight is 1.
    % Given clauses 0.

    1 p & -d # label(non_clause).  [assumption].
    2 q & -p # label(non_clause).  [assumption].
    5 p.  [clausify(1)].
    8 -p.  [clausify(2)].
    9 $F.  [copy(8),unit_del(a,5)].

    ====================== end of proof =========================

    ======================= PROOF ===============================

    % -------- Comments from original proof --------
    % Proof 2 at 0.00 (+ 0.00) seconds.
    % Length of proof is 5.
    % Level of proof is 2.
    % Maximum clause weight is 1.
    % Given clauses 0.

    1 p & -d # label(non_clause).  [assumption].
    3 -d # label(non_clause) # label(goal).  [goal].
    6 -d.  [clausify(1)].
    10 d.  [deny(3)].
    11 $F.  [copy(10),unit_del(a,6)].

    =================== end of proof ============================

    ======================= PROOF ===============================

    % -------- Comments from original proof --------
    % Proof 3 at 0.00 (+ 0.00) seconds.
    % Length of proof is 6.
    % Level of proof is 2.
    % Maximum clause weight is 2.
    % Given clauses 0.

    1 p & -d # label(non_clause).  [assumption].
    2 q & -p # label(non_clause).  [assumption].
    4 -p | -q.  [assumption].
    5 p.  [clausify(1)].
    7 q.  [clausify(2)].
    12 $F.  [back_unit_del(4),unit_del(a,5),unit_del(b,7)].

    ===================== end of proof ==========================
    \end{verbatim}
    \end{description}
    \item
    \begin{description}
    \item[(a)]Given set X={1,2,3}, all the Equivalent Relations on X.
    \begin{itemize}
        \item {(1, 1),(2, 2),(3, 3)}\\
          Therefore  X is Reflexive
        \item {(1, 1),(2, 2),(3, 3),(1, 2),(2, 1)}, {(1, 1),(2, 2),(3, 3),(2, 3),(3, 2)}, {(1, 1),(2, 2),(3, 3),(1, 3),(3, 1)}\\
        Therefore  X is Symmetric
        \item {(1, 1),(2, 2),(3, 3),(1, 2),(2, 1),(1, 3),(3, 1),(2, 3),(3, 2)}\\
        Therefore  X is Transitive.
    \end{itemize}
    \item[(b)] $Given\\
    (a,c ) \in R\\
    (b,c) \in R  -> (c,b) \in R (symmetry)\\
    if  R \in ER  and (a,c ) \in R, (b,c) \in R  -> (c,b) (symmetry)\\
    therefore (a,b) \in R (transitivity)$.
    
    \item[(c)]Let S = {1,2,3} and let R = {(1,2),(2,3),(1,3)}. Then R is not reflexive since (s,s) is  not in R for every element s of S, and R is not symmetric since (b,a) is not in R whenever (a,b) is in R. However, R is transitive, since (1,2), (2,3) and (1,3) are R.
    
    \item[(d)]Let S = {1,2,3} and let R = {(1,1), (2,2), (3,3), (1,2), (2,1), (2,3), (3,2)}. Then R is reflexive since (s,s) is in R for every element s of S, and R is symmetric since (b,a) is in R whenever (a,b) is in R. However, R is not transitive, since (1,2) and (2,3) are in R but (1,3) is not.

    
    
    \end{description}
\end{enumerate}

\end{document}
