
\documentclass[fullpage]{article}
\usepackage{graphicx}

% THIS FILE HAS TO BE RUN WITH PDFLATEX. IF YOU WANT TO RUN IT WITH LATEX AND YOU HAVE THE FIGURE IN eps, YOU CAN
% UNCOMMENT THE LINES BELOW. HOWEVER, IT IS EASIER TO PRODUCE FIGURES IN PDF: YOU CAN SEND ALMOST ANYTHING TO A THE "PDF PRINTER"
%\usepackage{epsfig}
%\usepackage{epstopdf}

\newcommand{\comlb}[1]{{\vspace{2mm}\noindent \bf REMARK (delete for submission):}~ #1 \hfill {\bf
    END.}\\}

\title{\vspace*{-2cm}\bf Assignment 3 Submission\\COMP 3400}
\author{Rehnuma Tarannum\\
Student Number: 100838870\\ Email: \  rehnumatarannum@cmail.carleton.ca}
\date{}
\begin{document}
\maketitle
\pagestyle{plain}
\thispagestyle{empty}

\noindent
\begin{enumerate}
\item 
\begin{description}
\item (a) I built my knowledge base with the following propositional variables
\begin{itemize}
\item a - Argentina is coloured Blue
\item b- Bolivia coloured Blue
\item c-Chile is coloured Blue
\item p -Peru coloured Blue
\item n - coloured not same color
\item s - coloured same color
\item d - share a border
\end{itemize}
\vspace{5mm}
\item (b) The following are my logical equations that depict the rules in trying to color the map with 2 colours.\\\\Equation (a) depicts that since Argentina, Bolivia and Peru share a border they cannot be coloured with same color.\\\\Equation (b) depicts that since Peru, Bolivia and Peru share a border they cannot be coloured with same color\\\\Equation (c) depicts that since Argentina and Peru do not share a border they can be colored with same colour\\\\Equation (d) depicts some condition when they countries are coloured same and share a border\\\\Equation (e) depicts some condition where countries are not same colour and have same border\\
\begin{enumerate}
\item
$ n \leftarrow a, b, c, d$.
\item 
$ n \leftarrow p, b, c, d$.
\item 
$ s \leftarrow a, p, \neg d$.
\item 
$ f \leftarrow s, d$.
\item 
$ g \leftarrow n, d$.
\end{enumerate}
\vspace{5mm}
\item [(c)] 
The main part of input file:
{\small\begin{verbatim}
:-dynamic(a/0).
:-dynamic(c/0).
:-dynamic(b/0).
:-dynamic(p/0).
:-dynamic(n/0).
:-dynamic(s/0).
:-dynamic(d/0).
:-dynamic(f/0).
:-dynamic(g/0).
n:- a, b, c, d.
n:- p, b, c, d.
s:- a, p, \+d.
f:- s, d.
g:- n, d.
\end{verbatim}}
\vspace{5mm}
The First Query:
{\small \begin{verbatim}
f.
\end{verbatim}}

The First Query result
{\footnotesize \begin{verbatim}
false
\end{verbatim}}
\vspace{5mm}
The Second Query:
{\small \begin{verbatim}
g.
\end{verbatim}}

The Second Query result
{\footnotesize \begin{verbatim}
false
\end{verbatim}}
\vspace{5mm}
Since both queries result in false then the system must be inconsistent hence I conclude that the map of these 4 countries cannot be coloured with 2 colours
\end{description}
\newpage
\item
\begin{description}
\item (a) I built my knowledge base with the following propositional variables and rules that mimics the problem statement
\begin{itemize}
    \item f - fly
    \item s - superheroes
    \item a - is abnormal(as superheroes) 
    \item b - is abnormal(in his own sense) 
    \item i - iron-man
    \item g - green lantern
    \item m - super-man
    \item c - wears flying suit
\end{itemize}
\vspace{5mm}
\begin{enumerate}
    \item 
    $f \leftarrow s, \neg a.$
    \item
    $s \leftarrow m.$
    \item
    $s \leftarrow g.$
    \item 
    $s \leftarrow i.$
    \item
    $a \leftarrow i, \neg b.$
    \item
    $b \leftarrow i, c.$
\end{enumerate}
\vspace{5mm}
(a) Superheroes fly if the are not abnormal(as superheroes)
\\\\(b) Superman is a superhero
\\\\(c) Green Lantern is a super hero
\\\\(d) Iron man is a superhero
\\\\(e) Iron man is abnormal unless he is abnormal(in his own sense)
\\\\(f) Iron man is abnormal(in his own sense) when he wears a flying suit\\
\newpage
\item (b)
\begin{enumerate}
    \item 
    $f \leftarrow s, \neg a.$
    \item
    $s \leftarrow m.$
    \item
    $s \leftarrow g.$
    \item 
    $s \leftarrow i.$
    \item
    $a \leftarrow i, \neg b.$
    \item
    $b \leftarrow i, c.$
    \item
    $s.$
\end{enumerate}
\vspace{5mm}

\begin{figure}[h]
\begin{center}
 \includegraphics[width=8cm]{tree1.png}
 \caption{The first refutation tree}\label{fig:sup}
\end{center}
\end{figure}
\newpage
\begin{figure}[h]
\begin{center}
 \includegraphics[width=12cm]{tree2.png}
 \caption{The second refutation tree}\label{fig:sup}
\end{center}
\end{figure}
\newpage
\begin{figure}[h]
\begin{center}
 \includegraphics[width=12cm]{tree3.png}
 \caption{The third refutation tree}\label{fig:sup}
\end{center}
\end{figure}

\newpage
\item (e)
The input for (b)
{\small\begin{verbatim}
:-dynamic(s/0).
:-dynamic(f/0).
:-dynamic(a/0).
:-dynamic(m/0).

:-dynamic(i/0).
:-dynamic(g/0).
:-dynamic(b/0).
:-dynamic(c/0).
f:- s, \+a.
s:- m.
s:- g.
s:- i. 
a:- i,\+b. 
b:- i, c.
s.
\end{verbatim}}
\vspace{5mm}
The input for (c)
{\small\begin{verbatim}
:-dynamic(s/0).
:-dynamic(f/0).
:-dynamic(a/0).
:-dynamic(m/0).

:-dynamic(i/0).
:-dynamic(g/0).
:-dynamic(b/0).
:-dynamic(c/0).
f:- s, \+a.
s:- m.
s:- g.
s:- i. 
a:- i,\+b. 
b:- i, c.
i.
\end{verbatim}}
\newpage
The input for (d)
{\small\begin{verbatim}
:-dynamic(s/0).
:-dynamic(f/0).
:-dynamic(a/0).
:-dynamic(m/0).

:-dynamic(i/0).
:-dynamic(g/0).
:-dynamic(b/0).
:-dynamic(c/0).
f:- s, \+a.
s:- m.
s:- g.
s:- i. 
a:- i,\+b. 
b:- i, c.
i.
c.
\end{verbatim}}
\vspace{5mm}
Query for all (b), (c), (d):
{\small \begin{verbatim}
f.
\end{verbatim}}
\vspace{5mm}
The Query result for (b)
{\footnotesize \begin{verbatim}
true
\end{verbatim}}
\vspace{5mm}
The Query result for (c)
{\footnotesize \begin{verbatim}
false
\end{verbatim}}
\vspace{5mm}
The Query result for (d)
{\footnotesize \begin{verbatim}
true
\end{verbatim}}


\end{description}
\end{enumerate}


\newpage
\section*{Appendix: Input/Output Files}
The path for 1 (c)
\vspace{5mm}
{\small\begin{verbatim}
Call:f
 Call:s
 Call:m
 Fail:m
 Redo:s
 Call:g
 Fail:g
 Redo:s
 Call:i
 Fail:i
 Redo:s
 Exit:s
 Call:a
 Call:i
 Fail:i
 Fail:a
 Redo:f
 Exit:f
 true
\end{verbatim}}
\vspace{5mm}
The path for 2 (b)
\vspace{5mm}
{\small\begin{verbatim}
Call:f
 Call:s
 Call:m
 Fail:m
 Redo:s
 Call:g
 Fail:g
 Redo:s
 Call:i
 Fail:i
 Redo:s
 Exit:s
 Call:a
 Call:i
 Fail:i
 Fail:a
 Redo:f
 Exit:f
 true
\end{verbatim}}

\vspace{5mm}
The path for 2 (c)
\vspace{5mm}
{\small\begin{verbatim}
 Call:f
 Call:s
 Call:m
 Fail:m
 Redo:s
 Call:g
 Fail:g
 Redo:s
 Call:i
 Exit:i
 Exit:s
 Call:a
 Call:i
 Exit:i
 Call:b
 Call:i
 Exit:i
 Call:c
 Fail:c
 Fail:b
 Redo:a
 Exit:a
 Fail:f
 false
\end{verbatim}}
\vspace{5mm}
The path for 2 (d)
\vspace{5mm}
{\small\begin{verbatim}
Call:f
 Call:s
 Call:m
 Fail:m
 Redo:s
 Call:g
 Fail:g
 Redo:s
 Call:i
 Exit:i
 Exit:s
 Call:a
 Call:i
 Exit:i
 Call:b
 Call:i
 Exit:i
 Call:c
 Exit:c
 Exit:b
 Fail:a
 Redo:f
 Exit:f
 true
\end{verbatim}}
\vspace{5mm}


\end{document} 